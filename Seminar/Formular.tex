\documentclass{article}
%\documentclass[smallextended]

\usepackage{graphicx}
\usepackage{xcolor} % added by Zhiyuan
\usepackage{amsfonts} % added by ZHiyuan
\usepackage{amsmath}
\usepackage{algorithm}
\usepackage{algorithmic}
\usepackage{multicol}
\usepackage{frame}
\usepackage{wrapfig}
\newcommand\vecPhi {\vec{\Phi}}
\begin{document}
\begin{align}
 \Phi \omega \approx R + \gamma  P^\pi  \Phi \omega\\ 
 Q^\pi = R + \gamma P^\pi Q^\pi \\
 Q(s, a) = \omega^T \phi (s, a) \\
 Q(s, a) = \sum_{i=1}^k \phi_i (s, a) \omega_i\\
 V^\pi(s) = \sum_a \pi(s, a) \sum_{s'} P_{ss'}^a [R_{ss'}^a + \gamma V^{\pi}(s')] \\
 \Phi^T ( \Phi - \gamma P^\pi \Phi) \omega^\pi = \Phi^T R \\
 \delta_t = \min_c \Vert \sum_{j=1}^{\vert Dic_{t-1} \vert} c_j\phi(x_j) - \phi(x) \Vert^2 \le \mu
\\
c_t = K_{t-1}^{-1} k_{t-1}(x) \\
\delta_t = k(x, x) - k_{t-1}^T(x) c_t
\end{align}

\end{document}